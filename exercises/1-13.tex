\documentclass{article}
\usepackage{amsmath}
\usepackage{pgfplots}
\pgfplotsset{compat=1.13}
\usepackage[utf8]{inputenc}

\title{SICP 1-13 exercise}
\author{Cyril Sadovnik }
\date{October 2016}

\begin{document}

\maketitle

\section{Exercise}

Prove that \(Fib(n)\) is the closest integer to \(\frac{\phi^n}{\sqrt 5}\), where \(\phi = \frac{1 + \sqrt 5}{2}\).
Hint: let \(\psi = \frac{1 - \sqrt 5}{2}\). Use induction and the defenition of the Fibonacci numbers to prove that \(Fib(n) = \frac{\phi^n - \psi^n}{\sqrt 5}\).

\section{Solutioin}
So we need to prove that \(Fib(n)\) is the closest integer to \(\frac{\phi^n}{\sqrt 5}\) by proving that \(Fib(n) = \frac{\phi^n - \psi^n}{\sqrt 5}\) using induction.

\subsection{Base step}

Let's check the formula for \(n = 1\):

\[ Fib(1) = \frac{\phi^1 - \psi^1}{\sqrt 5}\]
\[ 1 = \frac{\phi^1 - \psi^1}{\sqrt 5}\]
\[ 1 = \frac{\frac{1 + \sqrt 5}{2} - \frac{1 - \sqrt 5}{2}}{\sqrt 5}\]
\[ 1 = \frac{\frac{1 + \sqrt 5 - 1 + \sqrt 5}{2}}{\sqrt 5}\]
\[ 1 = \frac{\frac{2 \sqrt 5}{2}}{\sqrt 5}\]
\[ 1 = \frac{\sqrt 5}{\sqrt 5}\]
\[ 1 = 1\]

\subsection{Induction step}

Assuming it works for \(n = k\) and \(n = k - 1\), it must work for \(n = k + 1\) also:
\[ Fib(k + 1) = \frac{\phi^{k + 1} - \psi^{k + 1}}{\sqrt 5} \]

By defenition of \(Fib\): \(Fib(k + 1) = Fib(k - 1) + Fib(k)\), so
\[ Fib(k + 1) = \frac{\phi^{k - 1} - \psi^{k - 1}}{\sqrt 5} + \frac{\phi^k - \psi^k}{\sqrt 5} \]

Let's evaluate the right part of this equation and see what we got.

\[ \frac{\phi^{k - 1} - \psi^{k - 1}}{\sqrt 5} + \frac{\phi^k - \psi^k}{\sqrt 5} = \]
\[ \frac{\phi^{k - 1} - \psi^{k - 1} + \phi^k - \psi^k}{\sqrt 5} = \]
\[ \frac{\phi^{k - 1} + \phi^k - \psi^k - \psi^{k - 1}}{\sqrt 5} = ... \]

Now let's stop for a sec and consider this property of \(\phi\) and \(\psi\) :

\[ \phi = \frac{1}{\phi} + 1 \]

Same is true for \(\psi\).

\[ \frac{\phi^{k + 1} (\phi^{-2} + \phi^{-1}) - \psi^{k + 1}(\psi^{-2} + \psi^{-1}))}{\sqrt 5} = \]
\[ \frac{\phi^{k + 1} \phi^{-1} (\phi^{-1} + 1) - \psi^{k + 1} \psi^{-1} (\psi^{-1} + 1))}{\sqrt 5} = \]
\[ \frac{\phi^{k + 1} \phi^{-1} (\phi^{-1} + 1) - \psi^{k + 1} \psi^{-1} (\psi^{-1} + 1))}{\sqrt 5} = \]
\[ \frac{\phi^{k + 1} \phi^{-1} (\frac{1}{\phi} + 1) - \psi^{k + 1} \psi^{-1} (\frac{1}{\psi} + 1))}{\sqrt 5} = \]

Here comes the fun part. Using the property mentioned above we can now reduce expressions in the brackets:

\[ \frac{\phi^{k + 1} \phi^{-1} (\phi) - \psi^{k + 1} \psi^{-1} (\psi)}{\sqrt 5} = \]
\[ \frac{\phi^{k + 1} \phi^{-1} (\phi) - \psi^{k + 1} \psi^{-1} (\psi)}{\sqrt 5} = \]
\[ \frac{\phi^{k + 1} - \psi^{k + 1}}{\sqrt 5} \]

\subsection{Original hypothesis proof}

Now we can go back to the original hypothesis: \(Fib(n)\) is the closest integer to \(\frac{\phi^n}{\sqrt 5}\), where \(\phi = \frac{1 + \sqrt 5}{2}\).

\( \frac{\phi^n}{\sqrt 5} \) differs from \( \frac{\phi^n - \psi^n}{\sqrt 5}\) by \( \frac{\psi^n}{\sqrt 5} \) and always an integer for \( n \in N \) so the prof is a matter of proving that \( - 1/2 < \frac{\psi^n}{\sqrt 5} < 1/2 \) where \( n \in N \).

Let's just analyze \( \frac{\psi^n}{\sqrt 5} \):

\begin{tikzpicture}
\begin{axis}[
    xlabel = $x$,
    ylabel = {$\frac{\psi^n}{\sqrt 5}$},
]
\addplot [
    color=red,
    domain=0:10,
    samples=20000
]
{ ((-0.61803398875)^x) / sqrt(5) };
\end{axis}
\end{tikzpicture}

It's a descending function and it's maximum value on \(N\) (including 0) is approximately \( 0.4472135955 \), at \(x = 0\).

\par
quod erat demonstrandum

\end{document}